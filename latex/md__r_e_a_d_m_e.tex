\section*{$\ast$$\ast$\+IC\+: Projeto Settlers of Catan$\ast$$\ast$}

\subsection*{Trabalho Realizado por\+: Rodrigo Pinheiro\+\_\+a21802488 e Tomás Franco\+\_\+a21803301.}

\paragraph*{\href{https://github.com/ThomasFranque/IC_SettlersOfJapun}{\texttt{ Repositório do github}}.}

\subsubsection*{Solução}


\begin{DoxyItemize}
\item No começo do trabalho houve uma tentativa de usar o Open\+GL, porém, foi posta de parte pois não havia tempo suficiente para o que queríamos.
\item Houve também, a tentativa da criação de um .c que lê-\/se um ficheiro .txt para a impressão do board, mas também foi posta de lado.
\item Quando se começou a mexer com o ficheiro .ini, este mostrou-\/se ser mais complexo do que o que esperavamos, por essa razão, em vez de trabalhar tudo na mesma coisa, o trabalho foi dividido pelos dois elementos do grupo para não se perder muito tempo.
\item Este projeto já contém algum nível de complexidade portanto a criação de um fluxograma foi essencial\+: 
\item Todas as funçoes de lógica estão num único ficheiro (game\+\_\+logic.\+c).
\item No lançamento dos dados foi também definido que seriam 2 valores aleatorios de 1 a 6 somados.
\item Não são usadas quaisqueres variaveis globais.
\item F\+A\+L\+TA A\+I\+N\+DA C\+O\+I\+S\+AS \subsubsection*{Referencias}
\end{DoxyItemize}

\subparagraph*{Bibliotecas usadas}


\begin{DoxyItemize}
\item As bibliotecas utilizadas foram o $<$stdio.\+h$>$, $<$time.\+h$>$, $<$stdlib.\+h$>$ e uma exterior \href{https://github.com/rxi/ini}{\texttt{ $<$ini.\+h$>$}}. \subparagraph*{Competencias}
\end{DoxyItemize}


\begin{DoxyItemize}
\item O \char`\"{}work flow\char`\"{} de trabalhar em equipa no mesmo projeto mas em partes separadas.
\item Melhor familiarizados com o G\+CC, C, git e doxygen.
\item Criação de um \char`\"{}\+Makefile\char`\"{} funcional.
\item Como jogar Settlers of Catan.
\item Que ficheiros .ini são dores de cabeça quando não estão corretamente formatados. \subparagraph*{Manual de Utilizador}
\end{DoxyItemize}


\begin{DoxyItemize}
\item Na consola, quando no diretorio do jogo, fazer o comando {\itshape \$make} para gerar uma build do jogo.
\item Para limpar essa build basta executar o comando {\itshape \$make clean}
\item Para iniciar o jogo basta executar o comando, com a build feita, {\itshape \$./\+Settlers\+Of\+Japun} \subparagraph*{Troca de Ideias}
\end{DoxyItemize}


\begin{DoxyItemize}
\item Brick = Brick Lumber = Log (for pun purposes) Wool = Wool Iron = Steel (also for pun purposes) Desert = Desert
\item A variavel dos locais podia ser assim\+: int locales \mbox{[}xdim$\ast$ydim\mbox{]} (sei que n dá assim, mas qd conseguirmos usar o ini talvez dê) depois, para vila do jogador 1 seria = 1, para cidade seria = 2, para o jogador 2, vila seria = 3 e cidade seria = 4 nada seria = 0 como default um board 2x2 ficaria por exemplo\+: \mbox{[}1,2,0,4\mbox{]} \mbox{[}vila\+P1, cidade\+P1, nada, cidade\+P2\mbox{]}
\item Preços\+: Aldeia\+: 1 Brick + 1 Log + 1 Grain + 1 Wool. Cidade\+: 1 Grain + 1 Grain + 3 Steel.
\item A troca de ideias não é muito extensa porque na resolução deste trabalho estivemos em chamada pela plataforma \href{https://discordapp.com}{\texttt{ Discord}} sempre que possível. 
\end{DoxyItemize}